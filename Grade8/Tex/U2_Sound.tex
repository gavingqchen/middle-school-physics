\documentclass[windows,csize4]{BHCexam}
%\documentclass[windows,csize4,answers]{BHCexam}

\usepackage{multicol} % 分栏
\usepackage{polynom} % 多项式除法
\pagestyle{fancy}
\fancyfoot[C]{\kaishu \small 第 \thepage 页 共 \pageref{lastpage} 页}
%\fancyhead[L]{\includegraphics[width=2cm]{qrcode.png}}
\title{因式分解 - 对称式和轮换式}
%\subtitle{数学文科试卷}
%\notice{满分150分, 120分钟完成, \\	允许使用计算器,答案一律写在答题纸上.}
%\author{Gavin Chen}
%\date{\today}
\usepackage{enumerate} % 编号
\usepackage{cases}
\usepackage{subfigure}
\usepackage{graphicx}

\begin{document}

\maketitle

\begin{groups}
    \group{声音的产生与传播}{}
    声音是如何产生的?
    \begin{itemize}
        \item 由物体振动产生。振动停止,声音停止。
        \item 声源:正在发声的物体。
        \item 固体、液体、气体都可以振动产生声音。
    \end{itemize}

    声音是如何传播的?
    \begin{itemize}
        \item 声音传播需要介质。例子:把闹钟放在密封空间抽空气,声音越来越轻。(真空不能传递声音)
        \item 声音在不同介质中的传播速度,一般来说,传播速度固体>液体>气体。
        \item 声音以波的形式传播,称之为声波。
    \end{itemize}

    人耳听到声音,通常是由空气传播,然后进入耳道。但是也可以用骨传导耳机。(据说贝多芬耳朵聋了之后就采用类似方法作曲)
    
    \group{声音的特征}{}
响度,音调,音色。



    \group{如何利用声音}{}
    超声检查


\end{groups}

\label{lastpage}
\end{document}